\documentclass[answers]{exam}
% passed answers argument in []
% to document class to display answers

\usepackage[]{exam-randomizechoices}
\setrandomizerseed{10} % change seed to randomize question order and answers.

\title{\textbf{Exam Example with \LaTeX \space [Answers]}}
\author{Mike Poirier}
\date{February 21, 2023}

\begin{document}

\maketitle

\section{Multiple Choice}

\begin{questions}

\question What is the capital of France?

\begin{randomizechoices}
\choice London
\correctchoice Paris
\choice Berlin
\choice Madrid
\end{randomizechoices}

\question Who is the current president of the United States?

\begin{randomizechoices}
\choice George Washington
\choice Abraham Lincoln
\choice John F. Kennedy
\correctchoice Joe Biden
\end{randomizechoices}

\question What is the capital of Spain?

\begin{randomizechoices}
\choice London
\choice Paris
\choice Berlin
\correctchoice Madrid
\end{randomizechoices}

\question Who was the first man to walk on the moon?

\begin{randomizechoices}
\correctchoice Neil Armstrong
\choice Buzz Aldrin
\choice Michael Collins
\choice Yuri Gagarin
\end{randomizechoices}

\question Which of these famous physicists invented time?

\begin{oneparchoices}
 \choice Stephen Hawking 
 \choice Albert Einstein
 \choice Emmy Noether
 \correctchoice This makes no sense
\end{oneparchoices}
    
\question Which of these famous physicists published a paper on Brownian Motion?

\begin{checkboxes}
 \choice Stephen Hawking 
 \correctchoice Albert Einstein
 \choice Emmy Noether
 \choice I don't know
\end{checkboxes}

\section{Short and Long Answer}

\question Given the equation \(x^n + y^n = z^n\) for \(x,y,z\) and \(n\) positive
integers. 
\begin{parts}
\part[10] For what values of \(n\) is the statement in the previous question true?
\vspace{\stretch{1}}

\part[10] For \(n=2\) there's a theorem with a special name. What's that name?
\vspace{\stretch{1}}

\part[10] What famous mathematician had an elegant proof for this theorem but there was
not enough space in the margin to write it down?
\vspace{\stretch{1}}

\end{parts}

\question[20] Compute \[\int_{0}^{\infty} \frac{\sin(x)}{x}\]

\vspace{\stretch{1}}

\end{questions}

\end{document}
